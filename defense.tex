%!TEX program = xelatex

% Name           : defense.tex
% Author         : Jarrett Keifer (jkeifer0@gmail.com)
% Version        : 0.4
% Created on     : 05.05.2013
% Last Edited on : 24.03.2014
% Copyright      : Copyright (c) 2013-2014 by Benjamin Weiss. All rights reserved.
% License        : This file may be distributed and/or modified under the
%                  GNU Public License.
% Description    : HSRM beamer theme demonstration. Also includes a short 
%                  Tutorial regarding the beamer class.

\documentclass[draft,compress]{beamer}

% if handout option for beamer, make 4-up landscape
\mode<handout>{%
  \usepackage{pgfpages}%
  \pgfpagesuselayout{4 on 1}[letterpaper,border shrink=5mm, landscape]}

%--------------------------------------------------------------------------
% Common packages
%--------------------------------------------------------------------------
\usepackage[australian, american]{babel}
\usepackage{datetime}
\setdefaultdate{\dateaustralian}

\usepackage[overload]{textcase} %provides capitalization protection via \NoCaseChanges tag

\usepackage{pgf}  % for plots
\usepackage{setspace}

% Reference Settings
% *************** Set Biblatex Style and Options ***************

\usepackage{csquotes}
\usepackage[authordate-trad, backend=biber, firstinits=true, maxcitenames=3, isbn=false, doi=false, eprint=false, shorthandfull, sorting=nyt, sortcites=true, ibidtracker=false, abbreviate=true]{biblatex-chicago}
\addbibresource[datatype=bibtex]{thesis.bib}

%Check for page range in postnotes to use a colon not comma
\renewcommand{\postnotedelim}{\iffieldpages{postnote}{\addcolon\space}{\addcomma\space}} 
\DeclareFieldFormat{postnote}{#1} 

%Fix date format in bib -- could be moved to external .lbx file (with comments; change DefineBib... to DeclareBib...)
%%%%\DeclareLanguageMapping{american}{american-dmy}
%%%%
%%%%\begin{filecontents}{american-dmy.lbx}
%%%%\ProvidesFile{american-dmy.lbx}[american localisation with dmydate format for long dates]
%%%%
%%%%\InheritBibliographyExtras{american}
\DefineBibliographyExtras{american}{%
  \protected\def\mkbibdatelong#1#2#3{%
    \iffieldundef{#3}
      {}
      {\stripzeros{\thefield{#3}}%
       \iffieldundef{#2}{}{\nobreakspace}}%
    \iffieldundef{#2}
      {}
      {\mkbibmonth{\thefield{#2}}%
       \iffieldundef{#1}{}{\space}}%
    \iffieldbibstring{#1}{\bibstring{\thefield{#1}}}{\stripzeros{\thefield{#1}}}}%
}
%%%%\InheritBibliographyStrings{american}
%%%%\endinput
%%%%\end{filecontents}

% *************** Make ref links whole ref in text *****************
%%%%\DeclareCiteCommand{\cite}
%%%%  {\usebibmacro{prenote}}
%%%%  {\usebibmacro{citeindex}%
%%%%   \printtext[bibhyperref]{\usebibmacro{cite}}}
%%%%  {\multicitedelim}
%%%%  {\usebibmacro{postnote}}
%%%%
%%%%\DeclareCiteCommand*{\cite}
%%%%  {\usebibmacro{prenote}}
%%%%  {\usebibmacro{citeindex}%
%%%%   \printtext[bibhyperref]{\usebibmacro{citeyear}}}
%%%%  {\multicitedelim}
%%%%  {\usebibmacro{postnote}}
%%%%
%%%%\DeclareCiteCommand{\parencite}[\mkbibparens]
%%%%  {\usebibmacro{prenote}}
%%%%  {\usebibmacro{citeindex}%
%%%%    \printtext[bibhyperref]{\usebibmacro{cite}}}
%%%%  {\multicitedelim}
%%%%  {\usebibmacro{postnote}}
%%%%
%%%%\DeclareCiteCommand*{\parencite}[\mkbibparens]
%%%%  {\usebibmacro{prenote}}
%%%%  {\usebibmacro{citeindex}%
%%%%    \printtext[bibhyperref]{\usebibmacro{citeyear}}}
%%%%  {\multicitedelim}
%%%%  {\usebibmacro{postnote}}
%%%%
%%%%\DeclareCiteCommand{\footcite}[\mkbibfootnote]
%%%%  {\usebibmacro{prenote}}
%%%%  {\usebibmacro{citeindex}%
%%%%  \printtext[bibhyperref]{ \usebibmacro{cite}}}
%%%%  {\multicitedelim}
%%%%  {\usebibmacro{postnote}}
%%%%
%%%%\DeclareCiteCommand{\footcitetext}[\mkbibfootnotetext]
%%%%  {\usebibmacro{prenote}}
%%%%  {\usebibmacro{citeindex}%
%%%%   \printtext[bibhyperref]{\usebibmacro{cite}}}
%%%%  {\multicitedelim}
%%%%  {\usebibmacro{postnote}}
%%%%
%%%%\DeclareCiteCommand{\textcite}
%%%%  {\boolfalse{cbx:parens}}
%%%%  {\usebibmacro{citeindex}%
%%%%   \printtext[bibhyperref]{\usebibmacro{textcite}}}
%%%%  {\ifbool{cbx:parens}
%%%%     {\bibcloseparen\global\boolfalse{cbx:parens}}
%%%%     {}%
%%%%   \multicitedelim}
%%%%  {\usebibmacro{textcite:postnote}}
  
%Set custom strings for url and date labels (available at and last accessed)
\DefineBibliographyStrings{american}{%
  url = {available at},
  urlseen = {last accessed}, 
}
%\DeclareFieldFormat{url}{\bibstring{url}\addcolon\space\url{#1}}  % removed "Available at:" 2014-09-01
\DeclareFieldFormat{urldate}{\mkbibparens{\bibstring{urlseen}\space{#1}}}

% Set url + date appearance in bib (add doi here if needed)
\renewbibmacro*{bib+doi+url}{%
  \usebibmacro{url+urldate}
  }
  

% Abbreviate edited by as eds.
\DeclareFieldFormat{editortype}{\mkbibparens{#1}}
  
%Remove Quotes around titles
\DeclareFieldFormat
  [article,inbook,incollection,inproceedings,patent,thesis,unpublished]
  {title}{#1\isdot}
  
\appto{\bibsetup}{\raggedright} %align left to prevent stretched urls (perhaps remove?)

% /end bib settings


% Tables
\usepackage{tabu}
\usepackage{tabularx,ragged2e}
\usepackage{booktabs}
\usepackage{multicol}

%% Default Packages (Remove if not needed)
\usepackage{graphicx}

% Listingserweiterung
\usepackage{listings}
\lstset{ %
language=[LaTeX]TeX,
basicstyle=\normalsize\ttfamily,
keywordstyle=,
numbers=left,
numberstyle=\tiny\ttfamily,
stepnumber=1,
showspaces=false,
showstringspaces=false,
showtabs=false,
breaklines=true,
frame=tb,
framerule=0.5pt,
tabsize=4,
framexleftmargin=0.5em,
framexrightmargin=0.5em,
xleftmargin=0.5em,
xrightmargin=0.5em
}

%--------------------------------------------------------------------------
% Load theme
%--------------------------------------------------------------------------
\usetheme[seriffonts,onlysubsections]{hsrm}
\setbeamercovered{transparent=50}

\usepackage{dtklogos} % must be loaded after theme
\usepackage{tikz}
\usetikzlibrary{mindmap,backgrounds}

\usepackage[overload]{textcase}

%--------------------------------------------------------------------------
% General presentation settings
%--------------------------------------------------------------------------
\title{Agricultural Classification of Multi-Temporal MODIS Imagery in Northwest Argentina Using Kansas Crop Phenologies}
\subtitle{}
\date{\formatdate{17}{9}{2014}}
\author{Jarrett Keifer}
\institute{Department of Geography}%Portland State University}

%--------------------------------------------------------------------------
% Notes settings
%--------------------------------------------------------------------------
\setbeameroption{show notes}

\begin{document}
%--------------------------------------------------------------------------
% Titlepage
%--------------------------------------------------------------------------

\maketitle


%--------------------------------------------------------------------------
% Content
%--------------------------------------------------------------------------

\section*{Research Questions}

\begin{frame}{Research Questions}

Can I...

\begin{itemize}
  \item<1-> develop a phenological classification toolset?
  \item<2-> extract crop signatures from Kansas data?
  \item<3-> classify an Argentina study area with the Kansas signatures?
\end{itemize}
\end{frame}


%--------------------------------------------------------------------------
% Table of contents
%--------------------------------------------------------------------------
\section*{Outline}
\begin{frame}{Outline}
	% hideallsubsections ist empfehlenswert für längere Präsentationen
	\singlespace\tableofcontents[hideallsubsections]
\end{frame}


%--------------------------------------------------------------------------
% Content
%--------------------------------------------------------------------------

% BACKGROUND
\section{Background}

\subsection{Deforestation in Argentina}
\begin{frame}{Deforestation in Argentina}
\visible<2->{\begin{itemize}
  \item<2-> 1998 to 2002: 940,000 ha deforested
  \item<3-> \Lightit~Ley de Bosques \Light~passed in 2007
  \begin{itemize}
    \item<4-> Classified red, yellow, and green areas
  \end{itemize}
\end{itemize}}
\end{frame}

\begin{frame}{Deforestation in Argentina}

\begin{table}
  \centering
  \caption{Deforestation in Argentina, 2006 to 2011}
  \label{table:deforestationAR}
  \begin{tabu}{X[1,c]X[0.8,c]}
  \toprule
  \Book{Time Period} & \Book{Hectares Deforested} \\
  \midrule
  2006 to \Lightit~Ley de Bosques \Light~(2007) & 573,296 \\
  Ley de Bosques to OTBN (2009) & 473,001 \\
  OTBN to 2011 & 459,108 \\
  \midrule
  \Book{Total} & 1,505,405 \\
  \bottomrule
  \end{tabu}
\end{table}

\end{frame}

\begin{frame}{Deforestation in Argentina}
\begin{itemize}
  \item Deforestation has remained extremely high
  \item The effect of the the \Lightit~Ley de Bosques \Light~has been questioned
\end{itemize}
\end{frame}


\subsection{Soy and its Effects}
\begin{frame}{Soy and its Effects}
\begin{itemize}
  \item Argentina's soybean cultivation has continually increased
  \begin{itemize}
    \item 5 million ha in 1993 to 19 million ha in 2011
  \end{itemize}
\end{itemize}
\end{frame}

\begin{frame}{Soy and its Effects}
\begin{itemize}
  \item<1-> Soy production highly mechanized
  \item<2-> Over 99 percent of Argentine soy is genetically modified
  \begin{itemize}
    \item<2-> Resistance to glyphosate = heavy pesticide use
  \end{itemize}
  %\vspace{\baselineskip}
  \item<3-> \alert<4->{Capital requirements cut out small producers}
\end{itemize}
\end{frame}

\begin{frame}{Soy and its Effects}
\begin{itemize}
  \item<1-> Prevailing perception that soy drives deforestation
  \item<2-> Deforestation research has neglected to analyze specific crop cover
\end{itemize}
\end{frame}

\begin{frame}{Goal 1}
\begin{block}{Goal}
  Develop a crop mapping toolset which is efficient\\and economical
\end{block}
\vspace{\baselineskip}
 \visible<2->{Why is this important?
\begin{itemize}
  \item<3-> Better understanding of the dynamics of deforestation
  \item<4-> More effective land management policies
\end{itemize}}
\end{frame}


\subsection{Vegetation Indicies}
\begin{frame}{Vegetation Indicies}
\begin{alertblock}{Problem}
  Must be able to classify crops by type
\end{alertblock}
\vspace{\baselineskip}
 \visible<2->{\begin{itemize}
  \item<2-> A Vegetation Index (VI) can help with crop identification
  \begin{itemize}
    \item<3-> Normalized Difference Vegetation Index (NDVI)
    %\item<3-> Enhanced Vegetation Index (EVI)
  \end{itemize}}
\end{itemize}
\end{frame}

\begin{frame}{Vegetation Indicies}
\begin{block}{NDVI}
  \begin{equation*}
    NDVI = \frac{\rho~_{NIR} - \rho~_{red}}{\rho~_{NIR} + \rho~_{red}}
  \end{equation*}
\end{block}
\vspace{\baselineskip}
 \visible<2->{\begin{itemize}
  \item<2-> is a ratioing index
  \item<3-> minimizes multiplicative noise
  \item<4-> has issues with non-linearity and additive noise
\end{itemize}}
\end{frame}


%\begin{frame}{Vegetation Indicies}
%\begin{block}{EVI}
%  \begin{equation*}
%    EVI = G\frac{\rho~_{NIR} - \rho~_{red}}{\rho~_{NIR} +  C_1\times\rho~_{red} - C_2 \times \rho~_{blue} + L}
%  \end{equation*}
%\end{block}
%\end{frame}
%  
%\begin{frame}{Vegetation Indicies}
%\begin{block}{MODIS EVI}
%  \begin{equation*}
%    EVI = 2.5\frac{\rho~_{NIR} - \rho~_{red}}{\rho~_{NIR} +  6.0\times\rho~_{red} - 7.5 \times \rho~_{blue} + 1.0}
%  \end{equation*}
%\end{block}
%\vspace{\baselineskip} 
%
%\begin{itemize}
%  \item<1-> input bands require atmospheric correction
%  \item<2-> no saturation in high biomass
%\end{itemize}
%\end{frame}

\begin{frame}{Vegetation Indicies}
\begin{alertblock}{Problem}
  Must be able to classify crops by type
\end{alertblock}
\vspace{\baselineskip}
\visible<2->{\begingroup
\setbeamercolor{block title}{bg=hsrmSec2CompDark}
\setbeamercolor{block body}{bg=hsrmSec2Comp}
\begin{block}{Questions}
\begin{itemize}
  \item What if two crops have similar VI values on a single date?
  \item How to determine the VI values of a crop in an image?
\end{itemize}
\end{block}
\endgroup}
\end{frame}

\begin{frame}{Vegetation Indicies}
\begingroup
\setbeamercolor{block title}{bg=hsrmSec2CompDark}
\setbeamercolor{block body}{bg=hsrmSec2Comp}
\begin{block}{Question}
What if two crops have similar VI values on a single date?
\end{block}
\endgroup
\visible<2->{\begin{exampleblock}{Answer}
Use imagery from multiple dates.
\end{exampleblock}}
\end{frame}

\begin{frame}{Time Series Images}
NASA's Moderate Resolution Imaging Spectroradiometer (MODIS) Sensor
\begin{itemize}
  \item<1-> Terra and Aqua satellites
  \item<2-> Each images the Earth once per day
  \item<3-> Composite 16-day NDVI imagery at 250-meter resolution
%  \begin{itemize}
%    \item<3-> Both NDVI and EVI data
%  \end{itemize}
\end{itemize}
\end{frame}


\begin{frame}{Time Series Images}
\begin{columns}[c]
\column{0.45\textwidth}
Time Series Image (TSI)
\begin{itemize}
  \item<+-> Each band is a 16-day VI composite
  \item<+-> Bands are sequential composites
  \item<+-> Contains enough bands to cover an entire growing season
\end{itemize}

\column{0.45\textwidth}

\end{columns}
\end{frame}

\subsection{Phenological Classification}
\begin{frame}{Phenological Classification}

\begin{figure}
  \centering
  \includegraphicscopyright[width=0.85\linewidth]{Graphics/wardlowCropSignatures.png}{\autocite[From][]{wardlow2005state-level}}
\end{figure}

\end{frame}


\subsection{Phenological Classification}
\begin{frame}{Phenological Classification}
\begingroup
\setbeamercolor{block title}{bg=hsrmSec2CompDark}
\setbeamercolor{block body}{bg=hsrmSec2Comp}
\begin{block}{Question}
  How to determine the VI values of a crop in an image?
\end{block}
\endgroup
\visible<2->{\begin{exampleblock}{Answer}
  Existing approaches require training sites.
\end{exampleblock}}
%\centering
%\pause\pause~\Book~But what if you don’t have training sites?
\end{frame}

\begin{frame}{Phenological Classification}
\begin{alertblock}{Problem}
  What if you don’t have training sites?
\end{alertblock}
%\vspace{\baselineskip}
%\centering
%\pause\pause~\Book~But what if you don’t have training sites?
\end{frame}


% STUDY AREAS
\section{Study Areas}

\subsection{Kansas}
\begin{frame}{}
	
\end{frame}


\subsection{Pellegrini}
\begin{frame}{Department of Pellegrini}
	
\end{frame}


% DATA AND METHODS
\section{Data and Methods}

\subsection{Datasets}


\subsection{Pellegrini Data Collection}


\subsection{Data Processing}


% RESULTS AND DISCUSSION
\section{Results and Discussion}

\subsection{Pellegrini Data}

\subsection{Elminating Kansas Mixels}

\subsection{Kansas Crop Signatures}

\subsection{Kansas Verification}

\subsection{Argentina Classification}


% CONCLUSION
\section{Concluding Remarks}



%\begin{frame}{Systemvoraussetzungen}
%	Um erfolgreich Präsentationen mit diesem Theme erstellen zu können, sind folgende Voraussetzungen vom System zu erfüllen:
%	\begin{itemize}
%		\item Zum Setzen der Folien muss XeTeX verwendet werden.
%		\item Neben einigen Standardpaketen müssen die Pakete \texttt{beamer}, \texttt{pgf} und \texttt{xcolor} installiert sein.
%		\item Die Schriften \quoted{Flama-Light}, \quoted{Flama-Book} und \quoted{Flama-Medium} sollten installiert sein. Alternativ: \quoted{Arial}\\\url{http://www.felicianotypefoundry.com/}
%	\end{itemize}
%\end{frame}
%
%\section{Tutorial}
%\begin{frame}[containsverbatim]{Grundaufbau des Dokuments}
%Der Grundaufbau ist einfach:
%\begin{lstlisting}
%\documentclass[compress]{beamer}
%% Theme laden
%\usetheme{hsrm}
%% Allgemeine Präsentationseinstellungen 
%\title{Titel der Präsentation}
%\subtitle{Untertitel der Präsentation}
%\author{Ihr Name}
%\institute{Studienbereich\\Hochschule {\Medium RheinMain}}
%\begin{document}
%% Folien
%\end{document}
%\end{lstlisting}
%\end{frame}
%
%\begin{frame}{Themeoptionen}
%Um die Darstellung der Präsentation anzupassen können die folgenden Optionen gewählt werden.
%\begin{table}[]
%	\begin{tabularx}{\linewidth}{l>{\raggedright}X}
%		\toprule
%		\textbf{Option}			& \textbf{Auswirkung} \tabularnewline
%		\midrule
%		\texttt{noflama}		& Falls Sie die Schrift Flama nicht besitzen können Sie mit dieser Option auf die Schrift Arial umschalten. \tabularnewline
%		\texttt{noserifmath}		& Formeln werden ebenfalls serifenlos gesetzt. \tabularnewline
%		\texttt{nosectionpages} & Die Sektionseinleitungsseiten werden ausgeblendet.\tabularnewline
%		\bottomrule
%	\end{tabularx}
%	\label{tab:options}
%\end{table}
%\end{frame}
%
%\begin{frame}{Primärfarben}									
%Alle Farben des Corporate Designs sind im Template hinterlegt.
%
%
%\begin{multicols}{2}
%
%\setbeamercolor{hsrmRedDemo}{fg=hsrmRed,bg=white}
%\begin{beamercolorbox}[wd=\linewidth,ht=2ex,dp=0.7ex]{hsrmRedDemo}
%	\texttt{hsrmRed}
%\end{beamercolorbox}
%\setbeamercolor{hsrmRedDarkDemo}{fg=hsrmRedDark,bg=white}
%\begin{beamercolorbox}[wd=\linewidth,ht=2ex,dp=0.7ex]{hsrmRedDarkDemo}
%	\texttt{hsrmRedDark}
%\end{beamercolorbox}
%\setbeamercolor{hsrmWarmGreyDarkDemo}{fg=hsrmWarmGreyDark,bg=white}
%\begin{beamercolorbox}[wd=\linewidth,ht=2ex,dp=0.7ex]{hsrmWarmGreyDarkDemo}
%	\texttt{hsrmWarmGreyDark}
%\end{beamercolorbox}
%\setbeamercolor{hsrmWarmGreyLightDemo}{fg=hsrmWarmGreyLight,bg=white}
%\begin{beamercolorbox}[wd=\linewidth,ht=2ex,dp=0.7ex]{hsrmWarmGreyLightDemo}
%	\texttt{hsrmWarmGreyLight}
%\end{beamercolorbox}
%
%\setbeamercolor{hsrmRedDemoBg}{fg=white,bg=hsrmRed}
%\begin{beamercolorbox}[wd=\linewidth,ht=2ex,leftskip=.5ex,dp=0.7ex]{hsrmRedDemoBg}
%	\texttt{hsrmRed}
%\end{beamercolorbox}
%\setbeamercolor{hsrmRedDarkDemoBg}{fg=white,bg=hsrmRedDark}
%\begin{beamercolorbox}[wd=\linewidth,ht=2ex,leftskip=.5ex,dp=0.7ex]{hsrmRedDarkDemoBg}
%	\texttt{hsrmRedDark}
%\end{beamercolorbox}
%\setbeamercolor{hsrmWarmGreyDarkDemo}{fg=white,bg=hsrmWarmGreyDark}
%\begin{beamercolorbox}[wd=\linewidth,ht=2ex,leftskip=.5ex,dp=0.7ex]{hsrmWarmGreyDarkDemo}
%	\texttt{hsrmWarmGreyDark}
%\end{beamercolorbox}
%\setbeamercolor{hsrmWarmGreyLightDemo}{fg=white,bg=hsrmWarmGreyLight}
%\begin{beamercolorbox}[wd=\linewidth,ht=2ex,leftskip=.5ex,dp=0.7ex]{hsrmWarmGreyLightDemo}
%	\texttt{hsrmWarmGreyLight}
%\end{beamercolorbox}
%
%\end{multicols}
%
%\end{frame}
%
%\begin{frame}{Sekundärfarben}
%\begin{multicols}{2}
%
%\setbeamercolor{hsrmSec1Demo}{fg=hsrmSec1,bg=white}
%\begin{beamercolorbox}[wd=\linewidth,ht=2ex,dp=0.7ex]{hsrmSec1Demo}
%	\texttt{hsrmSec1}
%\end{beamercolorbox}
%\setbeamercolor{hsrmSec1DarkDemo}{fg=hsrmSec1Dark,bg=white}
%\begin{beamercolorbox}[wd=\linewidth,ht=2ex,dp=0.7ex]{hsrmSec1DarkDemo}
%	\texttt{hsrmSec1Dark}
%\end{beamercolorbox}
%\setbeamercolor{hsrmSec1CompDemo}{fg=hsrmSec1Comp,bg=white}
%\begin{beamercolorbox}[wd=\linewidth,ht=2ex,dp=0.7ex]{hsrmSec1CompDemo}
%	\texttt{hsrmSec1Comp}
%\end{beamercolorbox}
%\setbeamercolor{hsrmSec1CompDarkDemo}{fg=hsrmSec1CompDark,bg=white}
%\begin{beamercolorbox}[wd=\linewidth,ht=2ex,dp=0.7ex]{hsrmSec1CompDarkDemo}
%	\texttt{hsrmSec1CompDark}
%\end{beamercolorbox}
%
%\setbeamercolor{hsrmSec2Demo}{fg=hsrmSec2,bg=white}
%\begin{beamercolorbox}[wd=\linewidth,ht=2ex,dp=0.7ex]{hsrmSec2Demo}
%	\texttt{hsrmSec2}
%\end{beamercolorbox}
%\setbeamercolor{hsrmSec2DarkDemo}{fg=hsrmSec2Dark,bg=white}
%\begin{beamercolorbox}[wd=\linewidth,ht=2ex,dp=0.7ex]{hsrmSec2DarkDemo}
%	\texttt{hsrmSec2Dark}
%\end{beamercolorbox}
%\setbeamercolor{hsrmSec2CompDemo}{fg=hsrmSec2Comp,bg=white}
%\begin{beamercolorbox}[wd=\linewidth,ht=2ex,dp=0.7ex]{hsrmSec2CompDemo}
%	\texttt{hsrmSec2Comp}
%\end{beamercolorbox}
%\setbeamercolor{hsrmSec2CompDarkDemo}{fg=hsrmSec2CompDark,bg=white}
%\begin{beamercolorbox}[wd=\linewidth,ht=2ex,dp=0.7ex]{hsrmSec2CompDarkDemo}
%	\texttt{hsrmSec2CompDark}
%\end{beamercolorbox}
%
%\setbeamercolor{hsrmSec3Demo}{fg=hsrmSec3,bg=white}
%\begin{beamercolorbox}[wd=\linewidth,ht=2ex,dp=0.7ex]{hsrmSec3Demo}
%	\texttt{hsrmSec3}
%\end{beamercolorbox}
%\setbeamercolor{hsrmSec3DarkDemo}{fg=hsrmSec3Dark,bg=white}
%\begin{beamercolorbox}[wd=\linewidth,ht=2ex,dp=0.7ex]{hsrmSec3DarkDemo}
%	\texttt{hsrmSec3Dark}
%\end{beamercolorbox}
%\setbeamercolor{hsrmSec3CompDemo}{fg=hsrmSec3Comp,bg=white}
%\begin{beamercolorbox}[wd=\linewidth,ht=2ex,dp=0.7ex]{hsrmSec3CompDemo}
%	\texttt{hsrmSec3Comp}
%\end{beamercolorbox}
%\setbeamercolor{hsrmSec3CompDarkDemo}{fg=hsrmSec3CompDark,bg=white}
%\begin{beamercolorbox}[wd=\linewidth,ht=2ex,dp=0.7ex]{hsrmSec3CompDarkDemo}
%	\texttt{hsrmSec3CompDark}
%\end{beamercolorbox}
%
%\setbeamercolor{hsrmSec1DemoBg}{fg=white,bg=hsrmSec1}
%\begin{beamercolorbox}[wd=\linewidth,ht=2ex,leftskip=.5ex,dp=0.7ex]{hsrmSec1DemoBg}
%	\texttt{hsrmSec1}
%\end{beamercolorbox}
%\setbeamercolor{hsrmSec1DarkDemoBg}{fg=white,bg=hsrmSec1Dark}
%\begin{beamercolorbox}[wd=\linewidth,ht=2ex,leftskip=.5ex,dp=0.7ex]{hsrmSec1DarkDemoBg}
%	\texttt{hsrmSec1Dark}
%\end{beamercolorbox}
%\setbeamercolor{hsrmSec1CompDemoBg}{fg=white,bg=hsrmSec1Comp}
%\begin{beamercolorbox}[wd=\linewidth,ht=2ex,leftskip=.5ex,dp=0.7ex]{hsrmSec1CompDemoBg}
%	\texttt{hsrmSec1Comp}
%\end{beamercolorbox}
%\setbeamercolor{hsrmSec1CompDarkDemoBg}{fg=white,bg=hsrmSec1CompDark}
%\begin{beamercolorbox}[wd=\linewidth,ht=2ex,leftskip=.5ex,dp=0.7ex]{hsrmSec1CompDarkDemoBg}
%	\texttt{hsrmSec1CompDark}
%\end{beamercolorbox}
%
%\setbeamercolor{hsrmSec2DemoBg}{fg=white,bg=hsrmSec2}
%\begin{beamercolorbox}[wd=\linewidth,ht=2ex,leftskip=.5ex,dp=0.7ex]{hsrmSec2DemoBg}
%	\texttt{hsrmSec2}
%\end{beamercolorbox}
%\setbeamercolor{hsrmSec2DarkDemoBg}{fg=white,bg=hsrmSec2Dark}
%\begin{beamercolorbox}[wd=\linewidth,ht=2ex,leftskip=.5ex,dp=0.7ex]{hsrmSec2DarkDemoBg}
%	\texttt{hsrmSec2Dark}
%\end{beamercolorbox}
%\setbeamercolor{hsrmSec2CompDemoBg}{fg=white,bg=hsrmSec2Comp}
%\begin{beamercolorbox}[wd=\linewidth,ht=2ex,leftskip=.5ex,dp=0.7ex]{hsrmSec2CompDemoBg}
%	\texttt{hsrmSec2Comp}
%\end{beamercolorbox}
%\setbeamercolor{hsrmSec2CompDarkDemoBg}{fg=white,bg=hsrmSec2CompDark}
%\begin{beamercolorbox}[wd=\linewidth,ht=2ex,leftskip=.5ex,dp=0.7ex]{hsrmSec2CompDarkDemoBg}
%	\texttt{hsrmSec2CompDark}
%\end{beamercolorbox}
%
%\setbeamercolor{hsrmSec3Demo}{fg=white,bg=hsrmSec3}
%\begin{beamercolorbox}[wd=\linewidth,ht=2ex,leftskip=.5ex,dp=0.7ex]{hsrmSec3Demo}
%	\texttt{hsrmSec3}
%\end{beamercolorbox}
%\setbeamercolor{hsrmSec3DarkDemo}{fg=white,bg=hsrmSec3Dark}
%\begin{beamercolorbox}[wd=\linewidth,ht=2ex,leftskip=.5ex,dp=0.7ex]{hsrmSec3DarkDemo}
%	\texttt{hsrmSec3Dark}
%\end{beamercolorbox}
%\setbeamercolor{hsrmSec3CompDemo}{fg=white,bg=hsrmSec3Comp}
%\begin{beamercolorbox}[wd=\linewidth,ht=2ex,leftskip=.5ex,dp=0.7ex]{hsrmSec3CompDemo}
%	\texttt{hsrmSec3Comp}
%\end{beamercolorbox}
%\setbeamercolor{hsrmSec3CompDarkDemo}{fg=white,bg=hsrmSec3CompDark}
%\begin{beamercolorbox}[wd=\linewidth,ht=2ex,leftskip=.5ex,dp=0.7ex]{hsrmSec3CompDarkDemo}
%	\texttt{hsrmSec3CompDark}
%\end{beamercolorbox}
%
%\end{multicols}
%\end{frame}
%
%\begin{frame}[containsverbatim]{Folienstruktur}
%Strukturiert wird in Beamer wie in \LaTeX\ üblich mittels \lstinline!section!, \lstinline!subsection!, usw. Für Folien ist die \lstinline!frame! Umgebung definiert.
%
%Der Folientitel kann direkt an die \lstinline!frame! Umgebung übergeben werden oder mittels \lstinline!\frametitle{Folientitel}! innerhalb der Umgebung gesetzt werden.
%\begin{lstlisting}
%\section{Meine Sektion}
%\subsection{Meine Subsektion}
%\begin{frame}
%\frametitle{Folientitel}
%% Folieninhalt
%\end{frame}
%\end{lstlisting}
%\end{frame}
%
%\begin{frame}[containsverbatim]{Titelseite und Inhaltsverzeichnis}
%Die Titelseite erzeugt man mit 
%\begin{lstlisting}
%\maketitle
%\end{lstlisting}
%Und das Inhaltsverzeichnis mit
%\begin{lstlisting}
%\begin{frame}{Gliederung}
%	\tableofcontents[hideallsubsections]
%\end{frame}
%\end{lstlisting}
%Die Option \lstinline!hideallsubsections! bietet sich bei längeren Präsentationen an, um das Inhaltsverzeichnis kompakt zu halten.
%\end{frame}
%
%\subsection{Aufzählungen}
%\begin{frame}[containsverbatim]{Aufzählungen}
%Aufzählungen sind mit der \lstinline!enumerate! und der \lstinline!itemize! Umgebung möglich.
%\begin{enumerate}
%	\item Punkt 1
%	\item Punkt 2
%	\begin{itemize}
%		\item Punkt 1
%		\item Punkt 2
%	\end{itemize}
%	\item Punkt 3
%\end{enumerate}
%\end{frame}
%
%\subsection{Hervorhebungen}
%\begin{frame}[containsverbatim]{Hervorhebungen}
%In der Beamer Klasse ist die Funktion \lstinline!\alert! definiert, um einzelne Wörter hervorzuheben. Beispiel:
%\begin{itemize}
%	\item \alert{hervorgehobener Text}
%\end{itemize}
%Zusätzlich sind im HSRM Theme noch \lstinline!\quoted! und \lstinline!\doublequoted! definiert, um die Anführungszeichen des Corporate Designs der Hochschule einfach im Zugriff zu haben. Beispiele:
%\begin{itemize}
%	\item[] \quoted{Einfache Anführungszeichen}
%	\item[] \doublequoted{Doppelte Anführungszeichen}
%\end{itemize}
%\end{frame}
%
%\subsection{Blockstrukturen}
%\begin{frame}[containsverbatim]{Einfacher Block mit Aufzählung}
%Zur Strukturierung sind in Beamer Blockumgebungen definiert.
%\begin{block}{Block mit einer Aufzählung}
%	\begin{itemize}
%		\item Punkt 1
%		\item Punkt 2
%	\end{itemize}
%\end{block}
%\begin{lstlisting}
%\begin{block}{Block mit einer Aufzählung}
%	\begin{itemize}
%		\item Punkt 1
%		\item Punkt 2
%	\end{itemize}
%\end{block}
%\end{lstlisting}
%\end{frame}
%
%\begin{frame}[containsverbatim]{Alert Block}
%\begin{alertblock}{Alert Block}
%	Ein Alert Block wird mit der ersten Primärfarbe eingefärbt.
%\end{alertblock}
%\begin{lstlisting}
%\begin{alertblock}{Alert Block}
%Ein Alert Block wird mit der ersten Primärfarbe eingefärbt.
%\end{alertblock}
%\end{lstlisting}
%\end{frame}
%
%\begin{frame}[containsverbatim]{Example Block}
%\begin{exampleblock}{Example Block}
%	Ein Example Block wird mit der ersten Sekundärfarbe eingefärbt.
%\end{exampleblock}
%\begin{lstlisting}
%\begin{exampleblock}{Example Block}
%Ein Example Block wird mit der ersten Sekundärfarbe eingefärbt.
%\end{exampleblock}
%\end{lstlisting}
%\end{frame}
%
%\begin{frame}[containsverbatim]{Block mit anderer Farbe}
%\begingroup
%\setbeamercolor{block title}{bg=hsrmSec2Dark}
%\setbeamercolor{block body}{bg=hsrmSec2}
%\begin{block}{Block mit anderer Farbe}
%	In diesem Block wird eine weitere Sekundärfarbe verwendet.
%\end{block}
%\endgroup
%\begin{lstlisting}
%\begingroup
%\setbeamercolor{block title}{bg=hsrmSec2Dark}
%\setbeamercolor{block body}{bg=hsrmSec2}
%\begin{block}{Block mit anderer Farbe}
%	In diesem Block wird ...
%\end{block}
%\endgroup
%\end{lstlisting}
%\end{frame}
%
%\section{Beispielfolien}
%\begin{frame}{Weitere Beispiele}
%Nachfolgend sind weitere Beispielfolien ohne zusätzliche Erläuterung angehängt.
%
%Schauen Sie einfach in den Quelltext, um zu sehen wie die Folien erstellt wurden.
%\end{frame}
%\subsection{Abbildungen}
%\begin{frame}{Foto mit Copyright}
%	\begin{figure}
%		\centering
%		\includegraphicscopyright[width=\linewidth]{photo.jpg}{Copyright by \href{http://netzlemming.deviantart.com/}{Netzlemming}, \href{http://creativecommons.org/licenses/by-nc/3.0/}{CC BY-NC 3.0 License}}
%	\end{figure}
%\end{frame}
%\begin{frame}{Plot mit Beschriftung}
%	\begin{figure}
%		\centering
%		\input{plot.tex}
%		\caption{LFE channel frequency spectrum}
%	\end{figure}
%\end{frame}
%
%\subsection{Tabellen}
%\begin{frame}{Tabelle}
%\begin{table}[]
%	\caption{Selection of window function and their properties}
%	\begin{tabular}[]{lrrr}
%		\toprule
%		\textbf{Window}			& \multicolumn{1}{c}{\textbf{First side lobe}}	
%		                    & \multicolumn{1}{c}{\textbf{3\,dB bandwidth}}
%		                    & \multicolumn{1}{c}{\textbf{Roll-off}} \\
%		\midrule
%		Rectangular				& 13.2\,dB	& 0.886\,Hz/bin	& 6\,dB/oct		\\[0.25em]
%		Triangular				& 26.4\,dB	& 1.276\,Hz/bin	& 12\,dB/oct	\\[0.25em]
%		Hann					& 31.0\,dB	& 1.442\,Hz/bin	& 18\,dB/oct	\\[0.25em]
%		Hamming					& 41.0\,dB	& 1.300\,Hz/bin	& 6\,dB/oct		\\
%		\bottomrule
%	\end{tabular}
%	\label{tab:WindowFunctions}
%\end{table}
%\end{frame}
%
%\subsection{Formeln}
%\begin{frame}{Formeln}
%\begin{block}{Fourierintegral}
%\begin{equation*}
%F(\textrm{j}\omega) = \int\limits_{-\infty}^{\infty} f(t)\cdot\textrm{e}^{-\textrm{j}\omega t} dt
%\end{equation*}
%\end{block}
%\begin{block}{Fakultät}
%\begin{equation*}
%	n! = 1\cdot 2 \cdot 3 \cdot\ldots\cdot n = \prod_{k=1}^n k
%\end{equation*}
%\end{block}
%\end{frame}
%
%\begin{frame}{Mindmap mit TikZ}
%\centering
%\begin{tikzpicture}[scale=0.88]
%	%\tikzset{every child/.append style={level distance=250}}
%	\path[mindmap,concept color=hsrmWarmGreyLight,text=hsrmWarmGreyDark]
%	node[concept] {\TeX}
%	[clockwise from=-30]
%	child[concept color=hsrmSec2Dark,text=white] { node[concept] {\textcolor{white}{\XeTeX}} }
%	child[concept color=hsrmSec1CompDark,text=white] { node[concept] {\ConTeXt} }
%	child[concept color=hsrmSec1Dark,text=white] { node[concept] {\LaTeX} };
%\end{tikzpicture}
%\end{frame}
%
%\subsection{Fußnoten}
%\begin{frame}{Fußnoten}
%	Lorem ipsum dolor sit amet, consetetur sadipscing elitr, sed diam nonumy eirmod tempor invidunt ut labore et dolore magna aliquyam erat, sed diam voluptua. At vero eos et accusam et justo duo dolores et ea rebum. Stet clita kasd gubergren, no sea takimata sanctus est Lorem ipsum dolor sit amet. Lorem \footnote[frame,1]{Lorem ipsum dolor sit amet} ipsum dolor sit amet, consetetur sadipscing elitr, sed diam nonumy eirmod tempor invidunt ut labore et dolore magna aliquyam erat, sed diam voluptua. At vero eos et accusam et justo duo dolores et ea rebum. Stet clita kasd gubergren, no sea takimata sanctus est Lorem ipsum dolor sit amet.
%\end{frame}
%
%\subsection{Notizen}
%\begin{frame}{Folie mit dazugehöriger Notizfolie}
%    Für das Publikum ist diese Folie.
%
%	Für ihre Präsentation bieten sich folgende Programme an:
%	\begin{itemize}
%		\item Splitshow (Mac OS X)\\\url{https://code.google.com/p/splitshow/}
%		\item pdf-presenter (Windows)\\\url{https://code.google.com/p/pdf-presenter/}
%	\end{itemize}
%\end{frame}
%
%\note{
%    Für Ihre Notizen zum Vortrag vewenden Sie diese Folie.
%    
%	Für ihre Präsentation bieten sich folgende Programme an:
%	\begin{itemize}
%		\item Splitshow (Mac OS X)\\\url{https://code.google.com/p/splitshow/}
%		\item pdf-presenter (Windows)\\\url{https://code.google.com/p/pdf-presenter/}
%	\end{itemize} 
%}
%
%\subsection{Spalten}
%\begin{frame}{Zwei Spalten}
%	\begin{multicols}{2}
%		Lorem ipsum dolor sit amet, consetetur sadipscing elitr, sed diam nonumy eirmod tempor invidunt ut labore et dolore magna aliquyam erat, sed diam voluptua. At vero eos et accusam et justo duo dolores et ea rebum. Stet clita kasd gubergren, no sea takimata sanctus est Lorem ipsum dolor sit amet.
%		\begin{itemize}
%        	\item ein Eintrag
%        	\item noch ein Eintrag
%		\end{itemize}
%	\end{multicols}
%\end{frame}
%
%\begin{frame}{Spaltenumbruch}
%	\begin{multicols}{2}
%		Lorem ipsum dolor sit amet, consetetur sadipscing elitr, sed diam nonumy eirmod tempor invidunt ut labore et dolore magna aliquyam erat, sed diam voluptua. At vero eos et accusam et justo duo dolores et ea rebum. Stet clita kasd gubergren, no sea takimata sanctus est Lorem ipsum dolor sit amet.
%		\columnbreak
%		\begin{itemize}
%        	\item ein Eintrag
%        	\item noch ein Eintrag
%		\end{itemize}
%	\end{multicols}
%\end{frame}
%
%\begin{frame}{Literaturverzeichnis}
%	\begin{thebibliography}{10}
%    
%	\beamertemplatebookbibitems
%	\bibitem{Oppenheim2009}
%	Alan~V.~Oppenheim
%	\newblock \doublequoted{Discrete-Time Signal Processing}
%	\newblock Prentice Hall Press, 2009
%
%	\beamertemplatearticlebibitems
%	\bibitem{EBU2011}
%	European~Broadcasting~Union
%	\newblock \doublequoted{Specification of the Broadcast Wave Format (BWF)}
%	\newblock 2011
%  \end{thebibliography}
%\end{frame}
%
%\section{Ausblick}
%\begin{frame}{Bekannte Fehler}
%	\begin{itemize}
%		\item Theme ist momentan noch in einer einzelnen sty-Datei. Diese sollte unterteilt werden in einzelne Dateien für Schrift, Farbe usw.
%	\end{itemize}
%\end{frame}
%
%\begin{frame}{Fragen, Anmerkungen, Kontakt}
%	Das HSRM Theme steht unter der \quoted{GNU Public License}. Es darf also weitergegeben und modifiziert werden, sofern die Lizenzart beibehalten wird.
%	
%	Für Fragen und Anmerkungen stehe ich gerne zur Verfügung.
%	\begin{itemize}
%		\item \url{Benjamin.Weiss@student.hs-rm.de}
%	\end{itemize}
%\end{frame}

\end{document}






